\documentclass[a4paper, 11pt]{article}

\usepackage{kotex} % Comment this out if you are not using Hangul
\usepackage{fullpage}
\usepackage{hyperref}
\usepackage{amsthm}
\usepackage[numbers,sort&compress]{natbib}

\theoremstyle{definition}
\newtheorem{exercise}{Exercise}

\begin{document}
%%% Header starts
\noindent{\large\textbf{IS-521 Activity Proposal}\hfill
                \textbf{우승원}} \\
         {\phantom{} \hfill \textbf{seungwonwoo}} \\
         {\phantom{} \hfill Due Date: April 15, 2017} \\
%%% Header ends

\section{Activity Overview}

Activity-proposal로 DNS-Spoofing에 대한 실습을 제안합니다. DNS Spoofing은 웹 주소를 입력해서 브라우저에 접속할 때 DNS Server에 해당 주소의 도메인에 대한 IP Address를 물어보는데, 공격자가 잘못된 IP를 DNS Server보다 먼저 알려주어서 잘못된 주소로 접속하게 하는 공격입니다. (기본적으로 이를 위한 환경이 제공되어야 할 것 같습니다.)

\section{Exercises}

\begin{exercise}

  첫 번째로 DNS Spoofing을 하기 전에 DNS Query를 받기 위해서는 ARP Spoofing이 선행되어야 합니다. 기본적으로 제공되는 Client를 대상으로 ARP Spoofing을 실행합니다.

\end{exercise}

\begin{exercise}

  두 번째로 ARP Spoofing을 성공적으로 했다면 Client가 웹 브라우저에 접속하는 시나리오를 생각해 DNS Spoofing을 실행합니다.

\end{exercise}

\begin{exercise}

  세 번째로 두 번째 진행했던 DNS Spoofing 과정에서의 잘못된 IP Address를 자신의 address로 지정해 간단한 웹서버를 구현해 Client가 해당 웹서버를 확인할 수 있도록 합니다.

\end{exercise}

\section{Expected Solutions}

  기본적인 spoofing 과정을 이해하고 그 속에서 어떤 packet들을 주고 받는지 이해할 수 있게 됩니다.

\bibliography{references}
\bibliographystyle{plainnat}

\end{document}
